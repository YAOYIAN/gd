\documentclass{beamer}%
\usepackage[T1]{fontenc}%
\usepackage[utf8]{inputenc}%
\usepackage{lmodern}%
\usepackage{textcomp}%
\usepackage{lastpage}%
\usepackage{ctex}%
%
\usetheme{Madrid}%
\usecolortheme{beaver}%
\usepackage{blkarray}%
%
\begin{document}%
\normalsize%
\title{本节课的演示文档}%
\subtitle{\textit{由人工智能生成}}%
\author{Yian Yao}%
\date{\today}%
\maketitle%
\begin{frame}%
\frametitle{目录}%
\tableofcontents%
\end{frame}%
\section{回顾上一节课的知识点}%
\subsection{这里是关于上次课程的回顾}%
\begin{frame}%
\frametitle{\textsc{回顾上一节课的知识点:}}%
\begin{itemize}%
\item{\bf{回顾上一节课的知识点:}}%
\begin{itemize}%
\textit{}%
\textbf{\textit{艾萨克}}%
\textit{·}%
\textbf{\textit{牛顿}}%
\textit{的第二运动}%
\textbf{\textit{定律}}%
\textit{是他在1687年提出的。}%
\textbf{\textit{物体}}%
\textit{加}%
\textbf{\textit{速度}}%
\textit{的}%
\textbf{\textit{大小}}%
\textit{与所施加的}%
\textbf{\textit{力}}%
\textit{成}%
\textbf{\textit{正比}}%
\textit{。加}%
\textbf{\textit{速度}}%
\textit{的}%
\textbf{\textit{方向}}%
\textit{与}%
\textbf{\textit{力}}%
\textit{的}%
\textbf{\textit{方向}}%
\textit{相同。}%
\textbf{\textit{牛顿}}%
\textit{运动}%
\textbf{\textit{定律}}%
\textit{与第一和第三}%
\textbf{\textit{定律}}%
\textit{一起被称为}%
\textbf{\textit{牛顿}}%
\textit{运动}%
\textbf{\textit{定律}}%
\textit{。}%
\end{itemize}%
\end{itemize}%
\end{frame}%
\section{节选部分的摘要}%
\begin{frame}%
\frametitle{\textsc{节选部分摘要的说明}}%
\begin{itemize}%
\item{\bf{经过分析,这篇本节课程分为5个部分。}}%
\begin{itemize}%
\item{静电学}%
\item{静止电荷之间的相互作用力}%
\item{电场强度}%
\item{磁场和电流的描述}%
\item{光的电磁理论}%
\end{itemize}%
\end{itemize}%
\end{frame}%
\begin{frame}%
\frametitle{\textsc{节选部分的摘要}}%
\begin{block}{静电学}%
\textit{}%
\textbf{\textit{静电学}}%
\textit{是一门研究}%
\textbf{\textit{电场}}%
\textit{对}%
\textbf{\textit{电荷作用规律}}%
\textit{的}%
\textbf{\textit{学科}}%
\textit{。}%
\textbf{\textit{电荷}}%
\textit{只有两种,}%
\textbf{\textit{正电荷}}%
\textit{和}%
\textbf{\textit{负电荷}}%
\textit{。同一种}%
\textbf{\textit{电荷}}%
\textit{相互排斥,而不同}%
\textbf{\textit{类型}}%
\textit{的}%
\textbf{\textit{电荷}}%
\textit{相互吸引。}%
\textbf{\textit{电荷}}%
\textit{遵循}%
\textbf{\textit{电荷}}%
\textit{守}%
\textbf{\textit{恒定律}}%
\textit{。}%
\end{block}%
\begin{alertblock}{静止电荷之间的相互作用力}%
\textit{静止}%
\textbf{\textit{电荷}}%
\textit{之间的相互作用力符合}%
\textbf{\textit{库仑定律}}%
\textit{。相同的}%
\textbf{\textit{电荷}}%
\textit{在}%
\textbf{\textit{真空}}%
\textit{中相互排斥和吸引。}%
\textbf{\textit{电荷}}%
\textit{产生的}%
\textbf{\textit{电场}}%
\textit{由}%
\textbf{\textit{电场强度}}%
\textit{来描述。相互作用力是通过}%
\textbf{\textit{电荷}}%
\textit{产生的}%
\textbf{\textit{电场}}%
\textit{的相互作用而产生的。}%
\end{alertblock}%
\end{frame}%
\subsection{静电学;静止电荷之间的相互作用力}%
\begin{frame}%
\frametitle{\textsc{节选部分的摘要}}%
\begin{block}{电场强度}%
\textit{}%
\textbf{\textit{电场强度}}%
\textit{遵循}%
\textbf{\textit{电场强度}}%
\textit{叠加的}%
\textbf{\textit{原理}}%
\textit{。}%
\textbf{\textit{导体}}%
\textit{内部有可移动的}%
\textbf{\textit{自由电荷}}%
\textit{,}%
\textbf{\textit{绝缘体}}%
\textit{内部有束缚}%
\textbf{\textit{电荷}}%
\textit{。在}%
\textbf{\textit{电场}}%
\textit{的作用下,}%
\textbf{\textit{导体}}%
\textit{内的}%
\textbf{\textit{自由电荷}}%
\textit{会移动。}%
\textbf{\textit{静电}}%
\textit{平衡的}%
\textbf{\textit{条件}}%
\textit{是}%
\textbf{\textit{导体}}%
\textit{内部的}%
\textbf{\textit{电场强度}}%
\textit{为零。}%
\end{block}%
\begin{alertblock}{磁场和电流的描述}%
\textit{}%
\textbf{\textit{磁场}}%
\textit{是}%
\textbf{\textit{磁场}}%
\textit{对}%
\textbf{\textit{电流}}%
\textit{施加的}%
\textbf{\textit{力}}%
\textit{。}%
\textbf{\textit{磁场}}%
\textit{由}%
\textbf{\textit{磁感}}%
\textit{应}%
\textbf{\textit{强度}}%
\textit{来描述。}%
\textbf{\textit{感应电流}}%
\textit{的}%
\textbf{\textit{方向}}%
\textit{可以通过}%
\textbf{\textit{法拉第定律}}%
\textit{来确定。}%
\textbf{\textit{电流}}%
\textit{是}%
\textbf{\textit{电流}}%
\textit{和}%
\textbf{\textit{磁场}}%
\textit{之间磁相互作用的}%
\textbf{\textit{结果}}%
\textit{。}%
\end{alertblock}%
\end{frame}%
\subsection{电场强度;磁场和电流的描述}%
\begin{frame}%
\frametitle{\textsc{节选部分的摘要}}%
\begin{block}{光的电磁理论}%
\textit{}%
\textbf{\textit{麦克斯韦方程组}}%
\textit{描述了}%
\textbf{\textit{电磁场}}%
\textit{通常遵循的}%
\textbf{\textit{定律}}%
\textit{。}%
\textbf{\textit{电磁波}}%
\textit{在}%
\textbf{\textit{真空}}%
\textit{中的}%
\textbf{\textit{速度}}%
\textit{等于光速。}%
\textbf{\textit{电磁波}}%
\textit{是一种}%
\textbf{\textit{电磁波}}%
\textit{。光的波动}%
\textbf{\textit{理论}}%
\textit{属于}%
\textbf{\textit{电磁理论}}%
\textit{的}%
\textbf{\textit{范畴}}%
\textit{。它可以用来解决}%
\textbf{\textit{宏观}}%
\textit{的}%
\textbf{\textit{电动力学问题}}%
\textit{。}%
\end{block}%
\subsection{光的电磁理论}%
\end{frame}%
\begin{frame}%
\Huge{\centerline{谢谢观看!}}%
\end{frame}%
\end{document}