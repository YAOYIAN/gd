\documentclass{article}%
\usepackage[T1]{fontenc}%
\usepackage[utf8]{inputenc}%
\usepackage{lmodern}%
\usepackage{textcomp}%
\usepackage{lastpage}%
\usepackage{geometry}%
\geometry{margin=0.7in}%
\usepackage{ctex}%
\usepackage{ragged2e}%
\usepackage{fancyhdr}%
%
\fancypagestyle{header}{%
\renewcommand{\headrulewidth}{0pt}%
\renewcommand{\footrulewidth}{0pt}%
\fancyhead{%
}%
\fancyfoot{%
}%
\fancyhead[L]{%
Date: %
\today%
}%
\fancyhead[C]{%
Beijing Normal University%
}%
\fancyhead[R]{%
Page \thepage\ of \pageref{LastPage}%
}%
\fancyfoot[L]{%
Left Footer%
}%
\fancyfoot[C]{%
Center Footer%
}%
\fancyfoot[R]{%
Right Footer%
}%
}%
%
\begin{document}%
\normalsize%
\pagestyle{header}%
\begin{minipage}{\textwidth}%
\centering%
\begin{Large}%
\textbf{本节课的学生版讲义}%
\end{Large}%
\linebreak%
\begin{large}%
\textit{由人工智能生成}%
\end{large}%
\end{minipage}%
\section{本节课的完整总结}%
\label{sec:AllSummary}%
\textit{电磁场是研究静电产生的}%
\textbf{\textit{电场}}%
\textit{以及}%
\textbf{\textit{电场}}%
\textit{对}%
\textbf{\textit{电荷作用规律}}%
\textit{的学科。只有两种类型的电荷,称为正电荷和负电荷。同一种电荷相互排斥,而不同类型的电荷相互吸引。空间中某一点的}%
\textbf{\textit{电场强度}}%
\textit{由作用在该点测试电荷上的}%
\textbf{\textit{电场}}%
\textit{力的正单位定义,}%
\textbf{\textit{电场强度遵循场}}%
\textit{强叠加原理。通常,根据导电性,物质可以分为两类:导体和绝缘体。导体内部存在可移动的自由电荷;绝缘体,也称为电介质,其体内只有束缚电荷。在}%
\textbf{\textit{电场}}%
\textit{的作用下,导体内的自由电荷会移动。当导体的成分和温度均匀时,实现静电平衡的条件是导体内部的}%
\textbf{\textit{电场强度}}%
\textit{处处等于零。基于这个条件,可以导出}%
\textbf{\textit{导体静电}}%
\textit{平衡的几个性质。静磁学是研究}%
\textbf{\textit{电流稳态}}%
\textit{时产生的磁场以及磁场对}%
\textbf{\textit{电流}}%
\textit{施加的力的学科。}%
\textbf{\textit{电流}}%
\textit{产生的磁场用磁}%
\textbf{\textit{感应强度}}%
\textit{来描述。也就是说,}%
\textbf{\textit{电流}}%
\textit{在周围空间中产生磁场,磁场对放置在内部的}%
\textbf{\textit{电流施加力}}%
\textit{。}%
\textbf{\textit{电流}}%
\textit{产生的磁场用}%
\textbf{\textit{感应强度}}%
\textit{来描述。电磁场是一门学科。当通过闭合}%
\textbf{\textit{导体线圈}}%
\textit{的磁通量发生变化时,在线圈上产生}%
\textbf{\textit{感应电流}}%
\textit{。}%
\textbf{\textit{感应电流}}%
\textit{的方向可以由伦茨定律确定。闭合线圈中的}%
\textbf{\textit{感应电流}}%
\textit{是感应电动势的}%
\textbf{\textit{结果}}%
\textit{,该定律遵循法拉第定律:闭合线圈上}%
\textbf{\textit{感应电动力}}%
\textit{的大小总是与通过线圈的磁通量的时间变化率成比例。麦克斯韦方程组描述了电磁场通常遵循的定律。它与物质的介质方程、洛伦兹力公式和电荷守恒定律相结合,可以从理论上解决各种宏观电动力学问题。从麦克斯韦方程组导出的一个}%
\textbf{\textit{重要结果}}%
\textit{是}%
\textbf{\textit{电磁波}}%
\textit{的存在,}%
\textbf{\textit{电磁波}}%
\textit{以}%
\textbf{\textit{电磁波}}%
\textit{的形式传播。}%
\textbf{\textit{电磁波}}%
\textit{在真空中的速度等于光速。}

%
\section{本节课各知识点的概要}%
\label{sec:}%
\textit{%
}

%
\begin{minipage}{0.95\textwidth}%
\subsection{静电学}%
\label{subsec:}%
%
\textbf{静电学}%
是一门研究%
\textbf{电场}%
对%
\textbf{电荷作用规律}%
的%
\textbf{学科}%
。%
\textbf{电荷}%
只有两种,%
\textbf{正电荷}%
和%
\textbf{负电荷}%
。同一种%
\textbf{电荷}%
相互排斥,而不同%
\textbf{类型}%
的%
\textbf{电荷}%
相互吸引。%
\textbf{电荷}%
遵循%
\textbf{电荷}%
守%
\textbf{恒定律}%
。

%
\subsection{静止电荷之间的相互作用力}%
\label{subsec:}%
静止%
\textbf{电荷}%
之间的相互作用力符合%
\textbf{库仑定律}%
。相同的%
\textbf{电荷}%
在%
\textbf{真空}%
中相互排斥和吸引。%
\textbf{电荷}%
产生的%
\textbf{电场}%
由%
\textbf{电场强度}%
来描述。相互作用力是通过%
\textbf{电荷}%
产生的%
\textbf{电场}%
的相互作用而产生的。

%
\subsection{电场强度}%
\label{subsec:}%
%
\textbf{电场强度}%
遵循%
\textbf{电场强度}%
叠加的%
\textbf{原理}%
。%
\textbf{导体}%
内部有可移动的%
\textbf{自由电荷}%
,%
\textbf{绝缘体}%
内部有束缚%
\textbf{电荷}%
。在%
\textbf{电场}%
的作用下,%
\textbf{导体}%
内的%
\textbf{自由电荷}%
会移动。%
\textbf{静电}%
平衡的%
\textbf{条件}%
是%
\textbf{导体}%
内部的%
\textbf{电场强度}%
为零。

%
\subsection{磁场和电流的描述}%
\label{subsec:}%
%
\textbf{磁场}%
是%
\textbf{磁场}%
对%
\textbf{电流}%
施加的%
\textbf{力}%
。%
\textbf{磁场}%
由%
\textbf{磁感}%
应%
\textbf{强度}%
来描述。%
\textbf{感应电流}%
的%
\textbf{方向}%
可以通过%
\textbf{法拉第定律}%
来确定。%
\textbf{电流}%
是%
\textbf{电流}%
和%
\textbf{磁场}%
之间磁相互作用的%
\textbf{结果}%
。

%
\subsection{光的电磁理论}%
\label{subsec:}%
%
\textbf{麦克斯韦方程组}%
描述了%
\textbf{电磁场}%
通常遵循的%
\textbf{定律}%
。%
\textbf{电磁波}%
在%
\textbf{真空}%
中的%
\textbf{速度}%
等于光速。%
\textbf{电磁波}%
是一种%
\textbf{电磁波}%
。光的波动%
\textbf{理论}%
属于%
\textbf{电磁理论}%
的%
\textbf{范畴}%
。它可以用来解决%
\textbf{宏观}%
的%
\textbf{电动力学问题}%
。

%
\end{minipage}%
\section{请把你的笔记写在这}%
\label{sec:}%

%
\end{document}